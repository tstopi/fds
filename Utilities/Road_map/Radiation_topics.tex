\documentclass[12pt]{article}

%\usepackage{graphicx}
%\usepackage{amsmath}
%\usepackage{amssymb}
%\usepackage{amsfonts}
\usepackage{sectsty}
\sectionfont{\small}

\title{Research Topics for FDS Radiation Solver}
\newcommand{\bx}{{\bf x}}

\begin{document}
\maketitle



\section{Problem A: Computation and tabulation of local $\kappa$ and $\sigma$}
\label{Problem_A}


Absorption and scattering coefficients are integrated over the droplet
size distribution in the beginning of the simulation and stored in
lookup tables. For a spray with median diameter of $d_m$, the
coefficients are 
\begin{eqnarray}
\kappa_{\lambda,d}(\bx) = N(\bx) \int\limits_{r=0}^{\infty}~f(r,d_{m}(\bx))~C_{a}(r,\lambda)\textrm{d}r\label{kappa}\\
\sigma_{\lambda,d}(\bx) = N(\bx) \int\limits_{r=0}^{\infty}~f(r,d_{m}(\bx))~C_{s}(r,\lambda)\textrm{d}r\label{sigma}
\end{eqnarray}
\noindent
where:
\begin{itemize}
\item $\bx$ is the location vector ;
\item $r$ is the droplet radius ;
\item $C_{a}$ and $C_{s}$ are absorption and scattering cross
sections, respectively, calculated from the Mie theory with Wiscombe
code [Wiscombe] during the initialization phase ; 
\item $f$ is a size distribution similar to that used to define a
polydisperse spray at the injection point ;
\item $N(\bx)$ is the droplet density in the grid cell. By definition, it is equal to the number of drops contained in the cell divided by the cell volume :
\begin{equation}
 N (\bx)    =  \frac{\sum_i n_i }{\delta x~\delta y~\delta z} \label{N}
\end{equation}
\end{itemize}
The integrals of equations~\ref{kappa} and \ref{sigma} are not computed during the simulation for each cell again and again. Instead, they are pre-computed
during the initialization phase for a range of $d_m$ values, and the local values during the simulation are found by table lookup using the local value
$d_m(\bx)$.

Traditionally, the absorption and scattering coefficients have been stored as ''efficiency factors'' instead
of mean cross sections. For that purpose, the integrand has been divided by the cross section area of the median drop size, and the
coefficient have been computed by multiplying the integrals with mean area $A_d(\bx)$ instead of $N(\bx)$
\begin{eqnarray}
\kappa_{\lambda,d}(\bx) = A_d(\bx) \int\limits_{r=0}^{\infty}\frac{f(r,d_{m}(\bx))~C_{a}(r,\lambda)}{\pi (d_m/2)^2}\textrm{d}r\\
\sigma_{\lambda,d}(\bx) = A_d(\bx) \int\limits_{r=0}^{\infty}\frac{f(r,d_{m}(\bx))~C_{s}(r,\lambda)}{\pi (d_m/2)^2}\textrm{d}r
\end{eqnarray}
The mean area has been computed as 
\begin{equation}
A_d(\bx)\approx \frac{\rho_d(\bx)}{2\rho_w d_m(\bx)/3}
\end{equation}
where $\rho_d$ is the local mass concentration of droplets and $\rho_w$ is the water density.

In FDS 5.5, $A_d$ is computed as a direct sum of droplet cross sectional areas
\begin{equation}
A_d(\bx)    =  \frac{\sum_i n_i \pi r_i^2 }{\delta x~\delta y~\delta z}
\end{equation}

In principle, it should be possible, and even desirable to compute the radiative coefficients directly using equations~\ref{kappa} and \ref{sigma}.
The droplet number density would then be computed from Eq.~\ref{N}. However, numerical experiments have shown that this leads to substantial over-prediction
of the radiation attenuation both in the large scale spray (BRE\_Spray) and and small scale water mist micro nozzles.
\newline

{\em Task: Find out why the use of local number density and actual absorption and scattering coefficients leads to different results than
using absorption and scattering efficiency factors with local droplet area.}


\section{Problem B: Generalized liquid properties}

Currently, the complex refractive index is only available for water. One possible topic would be the inclusion of radiative properties 
other liquids than water, and generalizing the algorithm in that sense. 

\section{Problem C: Error analysis of wavelength limits}

In spray-radiation interaction model, the integration limits for the wavelenght integration are 1 and 200 $\mu$~m. For the gas phase properties,
the tails of the gray body radiation (soot) are taken into account, but not for sprays. \newline

{\em How big is the error caused by the 1 micron lower integration limit for hottest practical fires?}

\section{Problem D: Computational efficiency for fine water mists}

Simulations of laboratory scale radiation attenuation in small water mist sprays has shown that we need to resolve the spray dynamics extremely 
well in order to get enough attenuation. This means hundred times more droplets per second than what we do usually. 
Currently it is not known if this is completely radiation or completely spray issue, or mixture of both. 
\newline

{\em Find computationally efficient way to get the accurate description of the fine water mist sprays.}

\section{Problem E: Effect of the varying shape of the droplet size distribution}

In the precalculation of mean absorption and scattering cross sections, we assume that the droplet size distribution is the same as what
was introduced on the droplet boundary condition. This is not true far away in the spray. 
\newline

{\em How big is the error caused by this assumption? How to characterize the size distributions? How to consider in the radiation solver?}

\section{Problem F: Sooty water}

In fires, the water droplets absorb soot (cleaning the smoke). Should this be taken into account? How?



\end{document}

