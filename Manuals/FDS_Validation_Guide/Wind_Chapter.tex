% !TEX root = FDS_Validation_Guide.tex

\chapter{Wind Engineering and Atmospheric Dispersion}

This chapter presents results of simulations of wind over structures and atmospheric dispersion, all involving a simplified atmospheric boundary layer model in FDS.

%Computational fluid dynamics (CFD) has the potential of replacing wind tunnel testing in many wind engineering applications. Validated CFD software could enable structural engineers and builders to calculate wind effects on buildings for which no aerodynamic information is currently available. However, the use of CFD for structural engineering applications is limited mainly because of its prohibitive computational resource requirements. This is due in part to the need to simulate the imperfect spatial coherence of the low-frequency turbulent fluctuations in the incoming atmospheric boundary layer (ABL) flow.  A methodology is needed that would remove this barrier. In addition, it is desirable to develop software capable of readily incorporating features specific to the simulation of aerodynamic effects on bluff bodies. For this reason FDS is being adapted for wind engineering applications.
%
%In a first phase of this research, FDS has been used to compare data obtained from University of Western Ontario (UWO) wind tunnel tests on a 1:100 scale model of the Texas Tech University Wind Engineering Field Research Laboratory building (prototype dimensions: 9.1 m ($L$) $\times$ 13.7 m ($B$) $\times$ 4 m ($H$)). Pressure taps or ports along four lines on the model are chosen for comparison with FDS output. FDS is used to simulate flow in the wind tunnel and compare pressures on the building at several locations for varying angles of attack. Mesh refinement is also varied in the numerical simulation. Overall, the results from the FDS simulations fit the experimental data well.
%



%\section{FDS Model of the Wind Tunnel Tests}
%
%The experimental wind tunnel model developed at UWO is shown in Fig.~1. After developing the overall computational domain for this Wind Tunnel in FDS, mesh characteristics were determined for two levels of mesh refinement: coarse and fine. FDS cases were created in order to simulate the UWO wind tunnel tests for various angles of attack at these two levels of mesh refinement. Pressure taps along four lines on the UWO building model, shown in Fig.~2, were chosen for comparison with FDS output. Lines of ``output devices or ports'' were created in FDS at the same locations and the experimental data were plotted against the FDS output. The experiments were conducted at various angles of attack on the model, but first the tests conducted at 180\si{\degree} and 270\si{\degree} were used for comparisons with simulated results. These angles simplify the simulations in FDS because oblique angles complicate the calculations at the boundaries of the computational domain. All of the FDS cases use the same computational domain, which is a 1.12 m $\times$ 0.56 m $\times$ 0.28 m space split into eight uniform meshes. This domain size was chosen because it is large enough that the walls do not affect the wind flow and small enough that the simulations are not too time consuming. The domain is shown with the FDS model in Fig.~3.
%
%\begin{figure}
%\begin{center}
%\includegraphics[height=4in]{FIGURES/Wind_Engineering/UWO_Test_7_closeup}
%\caption[Close-up view of 1:100 scale WERFL building, Test 7]{Close-up view of 1:100 scale WERFL building, Test 7.}
%\label{UWO_Test_7_closeup}
%\end{center}
%\end{figure}
%
%%\begin{figure}
%%\begin{center}
%%\includegraphics[height=4in]{FIGURES/Wind_Engineering/pressure_tap_diagram_180_270}
%%\caption[Pressure tap positions, 180\si{\degree} and 270\si{\degree} cases]{Positions of pressure taps used in 180\si{\degree} and 270\si{\degree} cases at model (1:100) scale. The origin of the coordinate axes is at ground level at the center of the building in both cases.}
%%\label{pressure_tap_positions}
%%\end{center}
%%\end{figure}
%
%The domain is composed of a different size of cubic cells for each level of mesh refinement. In the coarser cases, each of the eight meshes is divided into 32 cells in each coordinate direction and each cell is a 0.00875 m cube. In the finer cases, the cells are half as large in each direction, yielding eight times as many cells. It was important to make the cells cubic in order to accurately represent turbulence in the flow simulation. This is because small scale turbulence can be resolved better with more accuracy and reliability when using cubic cells. Furthermore, non-cubic cells increase the likelihood of discretization error thereby enhancing the possibility of not resolving small scale turbulence correctly. One consequence of using cubic cells is that the size of the building model had to be slightly modified to be a multiple of the cell size in each direction, but accurate flow calculations were more important to this simulation than slight variations in the building size. These variations could be reduced by further mesh refinement.
%
%For the case of oblique angle flows, the computational domain was slightly modified. The domain was widened so that it was square in the $x$ and $y$ dimensions (ranging from -0.56 to 0.56 m in $x$ and $y$), two of the boundaries were set as inlet vents, and the two opposite boundaries were set to ``open'' condition. The same mesh resolution was used after widening the domain, so the whole domain was expanded to 16 computational meshes instead of 8 and the simulations became more computationally expensive as a result.
%
%\subsection{Input of Experimental Data into FDS Model}
%
%\paragraph{Inlet Velocity}
%The incoming flow assigned at the inflow boundary was characterized by its mean wind speed profile and turbulent fluctuations. For consistency with the UWO experiments, the mean wind speed profile was modeled by the power law:
%\begin{equation}
%U(z) = 9.144 \left(\frac{z}{0.0396}\right)^p
%\end{equation}
%where the 9.144 m/s is the reference wind speed at the 0.0396 m roof height of the model and the power law exponent $p=0.1173$ was fit to the experimental data.
%
%\paragraph{Turbulence Parameters}
%
%The synthetic turbulence parameters were set for the inlet surface. The fluctuations were generated by the Synthetic Eddy Method (SEM) based on the work of Jarrin [3]. The desired value of {\ct N\_EDDY} (the number of eddies generated at the inlet) was the largest possible value that would not slow down the simulation. This value was determined by running a number of simulations for the same time while varying the {\ct N\_EDDY} input and recording the total run time. The other turbulence parameters, {\ct L\_EDDY} (characteristic eddy length) and {\ct REYNOLDS\_STRESS}, were set based on the turbulence intensity data from the UWO experiments. The Reynolds stresses were found by multiplying the turbulence intensity in each direction by the wind velocity at roof height to determine the root mean square of the velocity fluctuations, and then squaring the root mean square values.
%
%\paragraph{Other Boundary Conditions}
%
%The side walls and roof were set to the free-slip condition in FDS. For the ground, the roughness length was specified in the experimental data to be 0.01 m at full scale. Therefore, the ground was set to a roughness value of 0.0001 m in the model. The building surface was set to the same conditions as the ground because setting a roughness length in FDS changes the wall stress model used in the calculations. Finally, the outlet of the wind tunnel was set to an ``open'' boundary condition, where pressure is set to the ambient pressure.
%
%After setting the boundary conditions, a test FDS case was created to confirm that the flow field in the simulation matched the desired inputs. Four planes of 42 ``output devices'' along the length of the domain were used to write out computational results. These planes are located just inside the inlet and at one quarter, one half, and three quarters of the total length. The devices output time series of the wind velocity in each coordinate direction throughout the simulation, and the resulting velocity data were used to determine the characteristics of the wind flow from the inlet as it varies in space and time. As expected, the velocity profile was found to match the inlet condition closely at each of the four planes.
%
%The variable (where is the friction velocity at the wall, is the distance to the wall, and is the kinematic viscosity of the fluid) is a non-dimensional wall distance for a wall-bounded flow. It was used as a measure of mesh refinement and was written in output boundary files to visualize the values on the surface of the building. The grid resolution in these FDS simulation cases was determined to be sufficient because the $y^+$ (a non-dimensional distance) values never exceeded 150 in the FDS simulation obtained using either the coarse or fine meshes. $y^+$ values are dependent on turbulence model wall laws and a value of 150 means that the first grid point is well within the logarithmic region. Four lines of devices were created on the building to measure and output values, and a boundary file was output to visualize the pressure coefficient. The devices were positioned on the building as shown in Fig.~3, and measured the mean and root mean square values of the pressure coefficient at each point over the simulation time. The pressure coefficient measurements were referenced to the velocity at roof height to be consistent with the experimental data. Overall, the results from the FDS simulations fit the experimental data well.
%
%
%\section{FDS Simulations}
%
%In this section, the FDS simulations are discussed first to ensure that the predicted results follow expected results obtained in typical wind tunnel experiments. Then, predicted FDS results are compared with available measured data obtained from wind tunnel experiments. FDS analyses were run for several angles of attack used in wind tunnel tests.
%
%%\subsection{Verification of FDS}
%%
%%One of the important features of the wind turbulence is its spectrum. The normalized reduced spectrum is defined as, where $n$ is frequency, $S$ is the spectral density of wind speed, $u$ is the mean wind speed, and  is the friction velocity that represents shear strength at the boundary layer [4]. The normalized spectrum is plotted against the normalized or reduced frequency defined as, also known as similarity coordinate (ratio of height to wavelength). A wind spectrum plot is shown in Fig.~4.
%%
%%Figure 4. Computed normalized wind spectrum vs. reduced frequency obtained with FDS simulation with fine mesh (643 cells per mesh).
%%
%%Points P1, P2, and P3 shown in Fig.~4 are located at roof height of the building in the middle of the computation domain (at position $L$/2 if $L$ is defined as the length of the domain). Similar spectra have been obtained for other angles of attack, and show similar trends. It can be seen that the empirical formula in Modified Kaimal Spectrum provides a good match with computed values in the high frequency range ($f>0.2$), as expected [4]. Such empirical formula are often conservatively valid in the high frequency range [4]. Another useful verification exercise was to plot the turbulence intensity along the length of the wind tunnel as a function of height. One such graph is shown in Fig.~5, which plots turbulence intensities in the $z$ direction at inlet, at $L$/4 ($L$=length of Wind tunnel), $L$/2, and 3$L$/4. The plot shows that the turbulence intensity reaches a maximum at a short distance from the floor for all four locations, after which it drops steadily with increase in height, as expected. Also, the turbulence intensity is maximum at the inlet, as should indeed be the case.
%%
%%
%%Figure 5. Computed turbulence intensity (in $z$-direction) along height at four locations along the length of the wind tunnel obtained using FDS simulation with fine mesh for wind angle of attack 135\si{\degree}.
%
%\subsection{Pressure Coefficient}
%
%Measured pressure coefficient data were obtained from the wind tunnel tests conducted at UWO. Four lines of ``output devices'' were created on the building (see Fig.~2) to output pressure coefficient values computed by FDS to compare with experimental measurements. The mean, maximum, minimum, and root mean square values of the pressure coefficient at each point (``output device'') were written by FDS over the entire simulation time. Again, the pressure coefficient measurements were referenced to the velocity at roof height to be consistent with the experimental data. Mean pressure coefficients (computed and experimental) at pressure taps located at a given y-coordinate for a 180\si{\degree} angle of attack are shown in Fig.~\ref{UWO_Test_7_pressure_coefficients_180}. Error bars were written using root mean squares of fluctuating velocity (e.g., one standard deviation).
%
%%The results show good agreement. Similar agreement was obtained for other angles of attack as well, including for wind directions not perpendicular to the faces of the building model). Note that FDS results show zero values at the bottom most point at the windward wall. This is because scalar quantities are assigned at the centroid of each cell in FDS, which is not at a height zero for this bottom most point on the windward wall.
%%
%%While the results were accurate for mean pressures, it was found that FDS do not match maximum and minimum pressures especially at the windward wall (see Figs.~7 (a) and (b)) The reason behind this is unclear at this point.
%
%%Apart from wind directions perpendicular to the face of the building, cornering flow cases were investigated for angles of attack of 30\si{\degree}, 35\si{\degree}, 40\si{\degree}, and 135\si{\degree} in FDS. The pressure coefficients plots show that, just like the perpendicular cases, in all cases FDS predicts mean values accurately but the maximum and minimum values are not matched well.
%
%\begin{figure}[p]
%\begin{tabular*}{\textwidth}{l@{\extracolsep{\fill}}r}
%\includegraphics[height=2.2in]{SCRIPT_FIGURES/Wind_Engineering/cp_mean_180_64_-0p0009y} &
%\includegraphics[height=2.2in]{SCRIPT_FIGURES/Wind_Engineering/cp_mean_180_64_-0p0296y} \\
%\includegraphics[height=2.2in]{SCRIPT_FIGURES/Wind_Engineering/cp_mean_180_64_-0p0437y} &
%\includegraphics[height=2.2in]{SCRIPT_FIGURES/Wind_Engineering/cp_mean_180_64_0p0256z}
%\end{tabular*}
%\caption[UWO Test 7 Pressure Coefficients, 180\si{\degree}]{UWO Test 7 mean pressure coefficients, 180\si{\degree} angle of attack.}
%\label{UWO_Test_7_pressure_coefficients_180}
%\end{figure}
%
%\begin{figure}[p]
%\begin{tabular*}{\textwidth}{l@{\extracolsep{\fill}}r}
%\includegraphics[height=2.2in]{SCRIPT_FIGURES/Wind_Engineering/cp_mean_270_64_-0p0067y} &
%\includegraphics[height=2.2in]{SCRIPT_FIGURES/Wind_Engineering/cp_mean_270_64_-0p0368y} \\
%\includegraphics[height=2.2in]{SCRIPT_FIGURES/Wind_Engineering/cp_mean_270_64_-0p0669y} &
%\includegraphics[height=2.2in]{SCRIPT_FIGURES/Wind_Engineering/cp_mean_270_64_0p0254z} \\
%\end{tabular*}
%\caption[UWO Test 7 Pressure Coefficients, 270\si{\degree}]{UWO Test 7 mean pressure coefficients, 270\si{\degree} angle of attack.}
%\label{UWO_Test_7_pressure_coefficients_270}
%\end{figure}

\section{LNG Dispersion Experiments}
\label{Atmospheric Dispersion}

The simulations of liquefied natural gas (LNG) dispersion experiments that are described in this section were originally designed by Jeffrey Engerer and Anay Luketa of Sandia National Laboratories on behalf of the Pipeline and Hazardous Materials Safety Administration of the U.S. Department of Transportation.

Parameters for the LNG dispersion experiments are given in Table~\ref{tab:LNG_Dispersion}. In some cases, values of the Monin-Obukhov parameters are taken directly from the test reports. However, for some of the experiments, these parameters were not derived using the same similarity functions as those presented above, in which case the parameters have been recomputed to best fit the measured velocity and temperature profiles. In the table, $u_*$ is the friction velocity, $\kappa=0.41$ is the Von K\'{a}rm\'{a}n constant, $z_0$ is the \emph{aerodynamic} roughness length, $\theta_*$ is the scaling potential temperature, $\theta_0$ is the ground level potential temperature, $L$ is the Monin-Obukhov length, and the similarity functions are those proposed by Dyer~\cite{Dyer:1974} and discussed in the report of the Falcon field experiments~\cite{Falcon}.

Figure~\ref{LNG_Dispersion_Burro_profiles} through Fig.~\ref{LNG_Dispersion_MaplinSands_profiles} display the measured velocity and temperature profiles, the corresponding Monin-Obukhov profiles that serve as initial and boundary conditions for FDS, and the resulting time-averaged profiles from the FDS simulations.

\subsubsection{Source Model}

In the various experiments, a fixed mass, $m$, of LNG was spilled onto water, forming a pool of increasing radius. For modeling purposes, it is assumed that the mass flux of natural gas from the circular pool is fixed at $\dot{m}''_{\rm max}=0.167$~kg/(m$^2 \cdot$s), and the temperature of the gas is $-162$~$^\circ$C, as suggested in the testing protocols. The diameter of the pool, $D$, is calculated using the assumed mass flux per unit area, the reported mass of LNG, $m$, and the spill duration, $\Delta t$.
\be
   D = \sqrt{ \frac{4 \, m}{\pi \, \dot{m}''_{\rm max} \, \Delta t} }
\ee
The values of $D$ are given in Table~\ref{tab:LNG_Dispersion}.



\subsubsection{Comparison of Measured and Predicted Downwind Gas Concentrations}

Figures~\ref{LNG_Dispersion_1}--\ref{LNG_Dispersion_2} compare measured and predicted downwind concentrations of natural gas originating from spills of liquefied natural gas (LNG) on water. In each case, the measured values are short-time (1~s to 3~s) averages of sensors positioned in arcs at discrete distances downwind of the spill site. For each arc, the maximum value is chosen. The processing of the FDS results follows the same procedure. The sensors were generally located a few meters off the relatively dry, flat terrain.


\begin{sidewaystable}[p]
\caption{Summary of LNG Dispersion Experiments.}
\begin{center}
\begin{tabular}{|l|c|c|c|c|c|c|c|c|c|c|c|c|c|}
\hline
Series                         & \multicolumn{4}{|c|}{Burro}       & \multicolumn{3}{|c|}{Coyote}  & \multicolumn{3}{|c|}{Falcon}                  & \multicolumn{3}{|c|}{Maplin Sands}          \\ \hline
Number                         & 3      & 7      & 8      & 9      &  3     & 5      & 6           &  1          & 3           & 4                 &  27             & 34             & 35      \\ \hline \hline
\multicolumn{14}{|c|}{Parameters supplied by test reports} \\ \hline
Fuel Mass, $m$ (kg)            & 14712  & 17289  & 12453  & 10730  & 6532   & 12676  & 10139       & 28074       & 21435       & 18984             & 3714            & 2094           & 3658    \\ \hline
Spill Duration, $\Delta t$ (s) & 167    & 174    & 107    & 79     & 65     & 98     & 82          & 131         & 154         & 301               & 160             & 95             & 135     \\ \hline
$p_0$ (mbar)                   & 948    & 940    & 941    & 940    & 936    & 939    & 942         & 908.9       & 900.8       & 906.3             & ---             & ---            & ---     \\ \hline
$T_0$ ($^\circ$C)              & 34.5   & 33.8   & 32.9   & 35.4   & 39.6   & 29.3   & 24.1        & 32.2        & 35.0        & 30.8              & 14.9            & 15.2           & 16.1    \\ \hline
RH (\%)                        & 5.2    & 7.4    & 4.5    & 14.4   & 11.3   & 22.1   & 22.8        & ---         & 4.0         & 12.0              & 53              & 90             & 77      \\ \hline
\multicolumn{14}{|c|}{Computed parameters} \\ \hline
$L$ (m)                        & -9.49  & -111   & 16.2   & -142   & -8.56  & -33.2  & 82.5        & 4.96        & -422        & 69.4              & -14.4           & -75.5          & -81.2   \\ \hline
$u_*$ (m/s)                    & 0.255  & 0.372  & 0.074  & 0.252  & 0.310  & 0.480  & 0.210       & 0.061       & 0.305       & 0.369             & 0.190           & 0.280          & 0.315   \\ \hline
$z_0$ (m)                      & 0.0002 & 0.0002 & 0.0002 & 0.0002 & 0.0002 & 0.0002 & 0.0002      & 0.008       & 0.008       & 0.008             & 0.0003$^\ddag$  & 0.0003$^\ddag$ & 0.0003$^\ddag$  \\ \hline
$\theta_*$ (K)                 & -0.532 & -0.097 & 0.026  & -0.035 & -0.890 & -0.520 & 0.039       & 0.058       & -0.018      & 0.152             & -0.180          & -0.075         & -0.088  \\ \hline
$\dot{q}''$ (W/m$^2$)          & -154   & -41    & 2      & -10    & -314   & -284   & 9           & 4           & -5          & 58                & -39             & -24            & -32     \\ \hline
$D$ (m)                        & 25.9   & 27.5   & 29.9   & 32.2   & 27.7   & 31.4   & 30.6        & 19.5$^\dag$ & 16.0$^\dag$ & 10.8$^\dag$       & 13.3            & 12.8           & 14.4    \\ \hline
\end{tabular}
\end{center}
$^\ddag$ The roughness length was changed to 0.00002~m to better match the measured velocity and temperature profiles

$^\dag$ The Falcon experiments involved 4 separated spills
\label{tab:LNG_Dispersion}
\end{sidewaystable}

\begin{figure}[p]
\begin{tabular*}{\textwidth}{l@{\extracolsep{\fill}}r}
\includegraphics[height=2.1in]{SCRIPT_FIGURES/LNG_Dispersion/Burro3_vel} &
\includegraphics[height=2.1in]{SCRIPT_FIGURES/LNG_Dispersion/Burro3_tmp} \\
\includegraphics[height=2.1in]{SCRIPT_FIGURES/LNG_Dispersion/Burro7_vel} &
\includegraphics[height=2.1in]{SCRIPT_FIGURES/LNG_Dispersion/Burro7_tmp} \\
\includegraphics[height=2.1in]{SCRIPT_FIGURES/LNG_Dispersion/Burro8_vel} &
\includegraphics[height=2.1in]{SCRIPT_FIGURES/LNG_Dispersion/Burro8_tmp} \\
\includegraphics[height=2.1in]{SCRIPT_FIGURES/LNG_Dispersion/Burro9_vel} &
\includegraphics[height=2.1in]{SCRIPT_FIGURES/LNG_Dispersion/Burro9_tmp}
\end{tabular*}
\caption[LNG Dispersion experiments, Burro velocity and temperature profiles]{LNG Dispersion experiments, Burro velocity and temperature profiles.}
\label{LNG_Dispersion_Burro_profiles}
\end{figure}

\begin{figure}[p]
\begin{tabular*}{\textwidth}{l@{\extracolsep{\fill}}r}
\includegraphics[height=2.1in]{SCRIPT_FIGURES/LNG_Dispersion/Coyote3_vel} &
\includegraphics[height=2.1in]{SCRIPT_FIGURES/LNG_Dispersion/Coyote3_tmp} \\
\includegraphics[height=2.1in]{SCRIPT_FIGURES/LNG_Dispersion/Coyote5_vel} &
\includegraphics[height=2.1in]{SCRIPT_FIGURES/LNG_Dispersion/Coyote5_tmp} \\
\includegraphics[height=2.1in]{SCRIPT_FIGURES/LNG_Dispersion/Coyote6_vel} &
\includegraphics[height=2.1in]{SCRIPT_FIGURES/LNG_Dispersion/Coyote6_tmp}
\end{tabular*}
\caption[LNG Dispersion experiments, Coyote velocity and temperature profiles]{LNG Dispersion experiments, Coyote velocity and temperature profiles.}
\label{LNG_Dispersion_Coyote_profiles}
\end{figure}

\begin{figure}[p]
\begin{tabular*}{\textwidth}{l@{\extracolsep{\fill}}r}
\includegraphics[height=2.1in]{SCRIPT_FIGURES/LNG_Dispersion/Falcon1_vel} &
\includegraphics[height=2.1in]{SCRIPT_FIGURES/LNG_Dispersion/Falcon1_tmp} \\
\includegraphics[height=2.1in]{SCRIPT_FIGURES/LNG_Dispersion/Falcon3_vel} &
\includegraphics[height=2.1in]{SCRIPT_FIGURES/LNG_Dispersion/Falcon3_tmp} \\
\includegraphics[height=2.1in]{SCRIPT_FIGURES/LNG_Dispersion/Falcon4_vel} &
\includegraphics[height=2.1in]{SCRIPT_FIGURES/LNG_Dispersion/Falcon4_tmp}
\end{tabular*}
\caption[LNG Dispersion experiments, Falcon velocity and temperature profiles]{LNG Dispersion experiments, Falcon velocity and temperature profiles.}
\label{LNG_Dispersion_Falcon_profiles}
\end{figure}

\begin{figure}[p]
\begin{tabular*}{\textwidth}{l@{\extracolsep{\fill}}r}
\includegraphics[height=2.1in]{SCRIPT_FIGURES/LNG_Dispersion/MaplinSands27_vel} &
\includegraphics[height=2.1in]{SCRIPT_FIGURES/LNG_Dispersion/MaplinSands27_tmp} \\
\includegraphics[height=2.1in]{SCRIPT_FIGURES/LNG_Dispersion/MaplinSands34_vel} &
\includegraphics[height=2.1in]{SCRIPT_FIGURES/LNG_Dispersion/MaplinSands34_tmp} \\
\includegraphics[height=2.1in]{SCRIPT_FIGURES/LNG_Dispersion/MaplinSands35_vel} &
\includegraphics[height=2.1in]{SCRIPT_FIGURES/LNG_Dispersion/MaplinSands35_tmp}
\end{tabular*}
\caption[LNG Dispersion experiments, Maplin Sands velocity and temperature profiles]{LNG Dispersion experiments, Maplin Sands velocity and temperature profiles.}
\label{LNG_Dispersion_MaplinSands_profiles}
\end{figure}


\begin{figure}[p]
\begin{tabular*}{\textwidth}{l@{\extracolsep{\fill}}r}
\includegraphics[height=2.1in]{SCRIPT_FIGURES/LNG_Dispersion/Burro3} &
\includegraphics[height=2.1in]{SCRIPT_FIGURES/LNG_Dispersion/Burro7} \\
\includegraphics[height=2.1in]{SCRIPT_FIGURES/LNG_Dispersion/Burro8} &
\includegraphics[height=2.1in]{SCRIPT_FIGURES/LNG_Dispersion/Burro9} \\
\includegraphics[height=2.1in]{SCRIPT_FIGURES/LNG_Dispersion/Coyote3} &
\includegraphics[height=2.1in]{SCRIPT_FIGURES/LNG_Dispersion/Coyote5} \\
\multicolumn{2}{c}{\includegraphics[height=2.1in]{SCRIPT_FIGURES/LNG_Dispersion/Coyote6}}
\end{tabular*}
\caption[LNG Dispersion experiments, Burro and Coyote]{LNG Dispersion experiments, Burro and Coyote.}
\label{LNG_Dispersion_1}
\end{figure}

\begin{figure}[p]
\begin{tabular*}{\textwidth}{l@{\extracolsep{\fill}}r}
\includegraphics[height=2.1in]{SCRIPT_FIGURES/LNG_Dispersion/Falcon1} &
\includegraphics[height=2.1in]{SCRIPT_FIGURES/LNG_Dispersion/Falcon3} \\
\multicolumn{2}{c}{\includegraphics[height=2.1in]{SCRIPT_FIGURES/LNG_Dispersion/Falcon4}} \\
\includegraphics[height=2.1in]{SCRIPT_FIGURES/LNG_Dispersion/MaplinSands27} &
\includegraphics[height=2.1in]{SCRIPT_FIGURES/LNG_Dispersion/MaplinSands34} \\
\multicolumn{2}{c}{\includegraphics[height=2.1in]{SCRIPT_FIGURES/LNG_Dispersion/MaplinSands35}}
\end{tabular*}
\caption[LNG Dispersion experiments, Falson and Maplin Sands]{LNG Dispersion experiments, Falcon and Maplin Sands.}
\label{LNG_Dispersion_2}
\end{figure}

\begin{figure}[p]
\begin{center}
\begin{tabular}{c}
\includegraphics[width=6.0in]{SCRIPT_FIGURES/ScatterPlots/FDS_Atmospheric_Dispersion}
\end{tabular}
\end{center}
\caption[Summary of LNG Dispersion predictions]{Summary of LNG Dispersion predictions. Note that the dashed black lines denote plus/minus a factor of two from the measured values. The red dashed lines represent two relative standard deviations about the solid red line, the model average.}
\label{Summary_LNG_Dispersion}
\end{figure}













